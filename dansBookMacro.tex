% additional latex packages to use
\usepackage{suffix}     % allows to add suffixes when defining commands
\usepackage{xifthen}    % for defining commands with optional arguments in a less opaque way

%%% GEOMETRY %%%%%%%%%%%%%%%%%%%%%%%%%%%%%%%%%%%%%%%%%%%%%%%%%%%%%%%%%%%%%%%%%%%
\usepackage[a4paper, inner=35mm, outer=25mm, top=25mm, bottom=25mm]{geometry}
\usepackage{pdflscape}  % for rotating pages in the PDF so you don't have to crane your neck to read them

%%% TEXT %%%%%%%%%%%%%%%%%%%%%%%%%%%%%%%%%%%%%%%%%%%%%%%%%%%%%%%%%%%%%%%%%%%%%%%
\usepackage{epigraph}   % for ``epigraphs'', i.e. rubrics at the start of chapters
\usepackage{csquotes}   % for fancy display-style quotes
\usepackage{xcolor}     % for changing colour of text, lines etc.
\usepackage{enumitem}   % allows for indented paragraphs within list env.
\setlist{listparindent=\parindent,parsep=0pt}
\usepackage{multicol}   % allows for multiple-columns within list environments

%%% GENERAL FLOATS %%%%%%%%%%%%%%%%%%%%%%%%%%%%%%%%%%%%%%%%%%%%%%%%%%%%%%%%%%%%%
\usepackage{afterpage}

%%% TABLES %%%%%%%%%%%%%%%%%%%%%%%%%%%%%%%%%%%%%%%%%%%%%%%%%%%%%%%%%%%%%%%%%%%%%
\usepackage[flushleft]{threeparttable}  % allows for notes at end of tables
\usepackage{tabularx}   % nice typesetting of tables, including line breaks
    \newcolumntype{L}{>{\raggedright\arraybackslash}X}

%%% GRAPHICS & FIGURES %%%%%%%%%%%%%%%%%%%%%%%%%%%%%%%%%%%%%%%%%%%%%%%%%%%%%%%%%
\usepackage{graphicx}
\graphicspath{ {images/} }
\usepackage{caption}
\usepackage{subcaption} % allows subfigures
\captionsetup{width=.9\textwidth,font=small,labelfont=small}
\captionsetup[subfigure]{justification=justified,singlelinecheck=false}

%%% CHAPTER & TOC SETUP %%%%%%%%%%%%%%%%%%%%%%%%%%%%%%%%%%%%%%%%%%%%%%%%%%%%%%%%
\usepackage{tocbibind}  % for lists in TOC
\usepackage[toc,page]{appendix} % for customizable appendices

\newcounter{lotdepth}   
\setcounter{lotdepth}{3}    % how deep does the list of tables go?

\newcounter{lofdepth}
\setcounter{lofdepth}{3}    % how deep does the list of figures go?

\usepackage{titlesec}
\usepackage{sectsty}    % to change format of all section and chapter headings

\newcommand*{\justifyheading}{\raggedright}

\titleformat    % custom chapter title format
{\chapter} % command
[display] % shape
{\sffamily\bfseries\Huge\justifyheading} % format
{\chaptertitlename \ \thechapter} % label
{0.5ex} % sep
{
    \LARGE
} % before-code
%[] % after-code

\allsectionsfont{\sffamily} % make all headings sans-serif

%%% PAGE SETUP; HEADER & FOOTER %%%%%%%%%%%%%%%%%%%%%%%%%%%%%%%%%%%%%%%%%%%%%%%%
\usepackage{fancyhdr}   % allows fancy formatting of headers & footers
\pagestyle{fancy}
\fancyhf{}
\fancyhead{}    % clear all header fields
\fancyfoot{}    % clear all footer fields
\fancyhead[LE]{\footnotesize\thepage}
\fancyhead[CE]{\footnotesize\selectfont\itshape\nouppercase{\ifnum\value{chapter}>0 \thechapter.~\fi\leftmark}}
\fancyhead[CO]{\footnotesize\selectfont\itshape\nouppercase{\ifnum\value{chapter}>0 \thechapter.~\fi\leftmark}}
\fancyhead[RO]{\footnotesize\thepage}
\renewcommand{\headrulewidth}{0pt}      % Line at the header invisible
\renewcommand{\footrulewidth}{0pt}      % Line at the footer invisible

\renewcommand{\chaptermark}[1]{\markboth{#1}{}}

\fancypagestyle{plain}{ % used for e.g. chapter title pages
    \fancyhf{}
    \fancyhead{}    % clear all header fields
    \fancyfoot{}    % clear all footer fields
    \fancyfoot[C]{\thepage}
    \renewcommand{\headrulewidth}{0pt}  % Line at the header invisible
    \renewcommand{\footrulewidth}{0pt}  % Line at the footer invisible
}

\fancypagestyle{frontAndBack}{ % for front and back matter
    \fancyhf{}
    \fancyhead{}    % clear all header fields
    \fancyfoot{}    % clear all footer fields
    \fancyhead[LE]{\footnotesize\thepage}
    \fancyhead[CE]{\footnotesize\selectfont\itshape\nouppercase{\ifnum\value{chapter}>0 \thechapter.~\fi\leftmark}}
    \fancyhead[CO]{\footnotesize\selectfont\itshape\nouppercase{\ifnum\value{chapter}>0 \thechapter.~\fi\leftmark}}
    \fancyhead[RO]{\footnotesize\thepage}
    \renewcommand{\headrulewidth}{0pt}      % Line at the header invisible
    \renewcommand{\footrulewidth}{0pt}      % Line at the footer invisible
}

\fancypagestyle{main}{ % for main matter (and appendices)
    \fancyhf{}
    \fancyhead{}    % clear all header fields
    \fancyfoot{}    % clear all footer fields
    \fancyhead[LE]{\footnotesize\thepage}
    \fancyhead[CE]{\footnotesize\selectfont\itshape\nouppercase{\ifnum\value{chapter}>0 \thechapter.~\fi\leftmark}}
    \fancyhead[CO]{\footnotesize\selectfont\itshape\nouppercase{\ifnum\value{chapter}>0 \thechapter.~\fi\leftmark}}
    \fancyhead[RO]{\footnotesize\thepage}
    \renewcommand{\headrulewidth}{0pt}      % Line at the header invisible
    \renewcommand{\footrulewidth}{0pt}      % Line at the footer invisible
}

%%% LINE SPACING %%%%%%%%%%%%%%%%%%%%%%%%%%%%%%%%%%%%%%%%%%%%%%%%%%%%%%%%%%%%%%%
\usepackage{setspace}
\setstretch{1.25}

%%% BIBLIOGRAPHY %%%%%%%%%%%%%%%%%%%%%%%%%%%%%%%%%%%%%%%%%%%%%%%%%%%%%%%%%%%%%%%
\usepackage[
    backref=true, 
    backend=biber, 
    bibencoding=utf8, 
    style=authoryear-comp, 
    maxcitenames=2, 
    uniquelist=false, 
    sorting=nyt,
    url=false,
    eprint=false,
    isbn=false,
    giveninits=true,
    uniquename=init
]{biblatex} % uniquelist=false means e.g. Siebesma et al. (2004) and Siebesma et
% al. (2007) do not have unique cite keys despite having different author lists;
% backref adds ref. in bib to page(s) cited in text
\renewbibmacro{in:}{
    \ifentrytype{article}{}{
        \printtext{\bibstring{in}\intitlepunct}
    }
} % don't print "in" for articles
\renewcommand*{\bibfont}{\small}

\AtEveryBibitem{
    \clearfield{day}
    \clearfield{month}
}   % stop superfluous fields from appearing

\DefineBibliographyStrings{english}{%
  backrefpage = {cit. on p.},% originally "cited on page"
  backrefpages = {cit. on pp.},% originally "cited on pages"
}

%%% REVIEW & EDITING %%%%%%%%%%%%%%%%%%%%%%%%%%%%%%%%%%%%%%%%%%%%%%%%%%%%%%%%%%%
\newboolean{draft}
\setboolean{draft}{true} % set to false in main doc preamble if you don't want annotations/to-do notes to appear
\usepackage{ulem} % required for strikethrough text
\normalem % use normal italics for emphasis rather than underline

% Display with annotation if draft == True; without annotation if False
\newcommand{\add}[1]{%
    \ifthenelse{ \boolean{draft} }%
    {\textcolor{blue}{#1}}%
    {\unskip#1}%
}
\newcommand{\delete}[1]{%
    \ifthenelse{ \boolean{draft} }%
    {\textcolor{red}{\textit{\sout{#1}}}}%
    {\unskip}%
}
\newcommand{\edit}[2]{%
    \ifthenelse{ \boolean{draft} }%
    {\textcolor{red}{\textit{\sout{#1}}} \textcolor{blue}{#2}}%
    {\unskip#2}%
}
\newcommand{\note}[1]{%
    \ifthenelse{ \boolean{draft} }%
    {\textcolor{red}{\textbf{#1}}}%
    {\unskip}%
}
\newcommand{\todonote}[1]{%
    \ifthenelse{ \boolean{draft} }%
    {\textcolor{red}{\textbf{[To-do: #1]}}}%
    {\unskip}%
}

%%% MATHEMATICS %%%%%%%%%%%%%%%%%%%%%%%%%%%%%%%%%%%%%%%%%%%%%%%%%%%%%%%%%%%%%%%%
%%% MATHS ENV. STUFF %%%%%%%%%%%%%%%%%%%%%%%%%%%%%%%%%%%%%%%%%%%%%%%%%%%%%%%%%%%
%%% PACKAGES
\usepackage{mathtools}
\usepackage{amsmath}
\usepackage{amssymb}
\usepackage{amsthm} % for theorem environments
\usepackage{amsfonts}   % specifically for blackboard bold
\usepackage{physics}    % package full of pre-defined physics notation, e.g. \abs and \norm for absolute value and norm, \vdot for vector dot product, \dv{}{} for derivatives, and \bra \ket \braket etc.
\usepackage{cancel} % for nice cancelling out of terms in equations

%%% COMMANDS
%% Note: some of these commands require the ifthen and/or suffix packages
% set {...}
\newcommand{\set}[1]{ \ensuremath{ \left\{ #1 \right\} } }
% ``Big Oh''
\newcommand{\bigOh}[1]{ \ensuremath{ \mathcal{O}\left( #1 \right) } }
% vector
\renewcommand{\vec}[1]{\ensuremath{\boldsymbol{\mathbf{#1}}}}
% vector \ell
\newcommand{\vecEll}{\ensuremath{\boldsymbol\ell}}
% unit vector
\newcommand{\unit}[1]{\ensuremath{\hat{\vec{#1}}}}
% nice version of measure dx for integrals
\newcommand{\diff}[2][]{%
    \ifthenelse{ \isempty{#1} }
        {\ensuremath{\operatorname{d}\!{#2}}}
        {\ensuremath{\operatorname{d}^{#1}\!{#2}}}
}
% horizontal divergence
\newcommand{\divH}[1]{\ensuremath{\boldsymbol{\nabla}_\text{H}\mathbin{\vdot} #1}}
% tensor
\newcommand{\tensor}[1]{\ensuremath{\boldsymbol{\mathsf{#1}}}}
% tensor parts
\newcommand{\iso}{\ensuremath{\tensor{iso}}}
\newcommand{\dev}{\ensuremath{\tensor{dev}}}
\newcommand{\symm}{\ensuremath{\tensor{symm}}}
\newcommand{\asymm}{\ensuremath{\tensor{asymm}}}
% double-colon contraction notation
\newcommand{\contract}{\ensuremath{\mathbin{\boldsymbol{:}}}}
% trace
\renewcommand{\tr}{\ensuremath{\tensor{tr}}}
% tensor identity
\newcommand{\ident}[1][]{%
    \ifthenelse{ \isempty{#1} }
        {\ensuremath{\tensor{I}}}
        {\ensuremath{\tensor{I}}_{#1}}
}
% transpose
\newcommand*{\trans}{\ensuremath{^{\mathsf{T}}}}
% big-D material derivative
\newcommand{\Dv}[2]{\ensuremath{\frac{\operatorname{D}\!{#1}}{\operatorname{D}\!{#2}}}}
\WithSuffix\newcommand\Dv*[2]{\ensuremath{\flatfrac{\operatorname{D}\!{#1}}{\operatorname{D}\!{#2}}}}
% multi-fluid material derivative
\newcommand{\DvMF}[3]{\ensuremath{\frac{\operatorname{D}_{#1}\!{#2}}{\operatorname{D}\!{#3}}}}
\WithSuffix\newcommand\DvMF*[3]{\ensuremath{\flatfrac{\operatorname{D}_{#1}\!{#2}}{\operatorname{D}\!{#3}}}}
% Laplacian and divergence commands consistent with \grad
\renewcommand{\laplacian}{\grad^2}
\newcommand{\laplaciantilde}{\widetilde{\grad}^2}
\newcommand{\laplacianhat}{\widehat{\grad}^2}
\newcommand{\divtilde}[1]{\widetilde{\grad}\mathbin{\vdot}{#1}}
\newcommand{\divhat}[1]{\widehat{\grad}\mathbin{\vdot}{#1}}
% nice filtering operator
\newcommand{\filter}[2][]{%
    \ifthenelse{ \isempty{#1} }
        {\ensuremath{ \left\langle {#2} \right\rangle }}
        {\ensuremath{ \left\langle {#2} \right\rangle_{#1} }}
}
% nice Reynolds averaging operator
\newcommand{\reynolds}[1]{%
        {\ensuremath{ \left\langle {#1} \right\rangle_{\mathrm{R}} }}
}
% resolved and subfilter parts
\newcommand{\resolved}[2][]{%
    \ifthenelse{ \isempty{#1} }
        {\ensuremath{ {#2}^\mathrm{r} }}
        {\ensuremath{ {#2}_{#1}^\mathrm{r} }}
}
\newcommand{\subfilter}[2][]{%
    \ifthenelse{ \isempty{#1} }
        {\ensuremath{ {#2}^\mathrm{s} }}
        {\ensuremath{ {#2}_{#1}^\mathrm{s} }}
}
\newcommand{\favreResolved}[2][]{%
    \ifthenelse{ \isempty{#1} }
        {\ensuremath{ \widetilde{#2}^\mathrm{r} }}
        {\ensuremath{ \widetilde{#2}_{#1}^\mathrm{r} }}
}
\newcommand{\favreSubfilter}[2][]{%
    \ifthenelse{ \isempty{#1} }
        {\ensuremath{ \widetilde{#2}^\mathrm{r} }}
        {\ensuremath{ \widetilde{#2}_{#1}^\mathrm{r} }}
}
% dimensionless groups for fluid dynamics
\renewcommand{\Re}{\ensuremath{\operatorname{Re}}}  % Reynolds number
\renewcommand{\Pr}{\ensuremath{\operatorname{Pr}}}  % Prandtl number
\newcommand{\Ra}{\ensuremath{\operatorname{Ra}}}    % Rayleigh number
\newcommand{\Nu}{\ensuremath{\operatorname{Nu}}}    % Nusselt number
\newcommand{\Co}{\ensuremath{\operatorname{Co}}}    % Courant number
\newcommand{\Fr}{\ensuremath{\operatorname{Fr}}}    % Froude number
\newcommand{\Ro}{\ensuremath{\operatorname{Ro}}}    % Rossby number

%%% THEOREM, REMARK, ASSUMPTION ETC. ENVIRONMENTS %%%%%%%%%%%%%%%%%%%%%%%%%%%%%%
\theoremstyle{definition}
\newtheorem{assumption}{Assumption}
\newtheorem{definition}{Definition}
\newtheorem*{conjecture}{Conjecture}
\theoremstyle{plain}
\newtheorem*{aside}{Aside}

%%% MISC. %%%%%%%%%%%%%%%%%%%%%%%%%%%%%%%%%%%%%%%%%%%%%%%%%%%%%%%%%%%%%%%%%%%%%%
% allow display equation environments to break over page breaks
\allowdisplaybreaks

%%% REFERENCES, URLS & HYPERLINKING %%%%%%%%%%%%%%%%%%%%%%%%%%%%%%%%%%%%%%%%%%%%
\PassOptionsToPackage{hyphens}{url}\usepackage{hyperref}   % everyone says to include hyperref last...
\hypersetup{
    colorlinks=false
}   % colour links looked pretty horrible...
\usepackage{xurl} % for nice line-breaks in URLs
\usepackage[capitalise,noabbrev]{cleveref}   % nice automatic typesetting of e.g. theorem, chapter etc. references
\AtBeginEnvironment{appendices}{\crefalias{chapter}{appendix}} % change chapter to section if you want within-chapter appendices